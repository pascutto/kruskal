\documentclass[handout]{beamer}

\usepackage[utf8]{inputenc}
\usepackage{algorithmic}
\usepackage{algorithm}
\usepackage[french]{babel}

\usetheme{Warsaw}

\title[Algorithme de Kruskal]{Algorithme de Kruskal\\Application au 
problème du voyageur de commerce}
\author{Mihai Dusmanu, Clément Pascutto}
\institute{Algorithmique et programation\\École normale supérieure}
\date{Vendredi 8 Janvier 2016}


\begin{document}

    \begin{frame}
		\titlepage
    \end{frame}

	\begin{frame}
		\frametitle{Sommaire}
		\tableofcontents
	\end{frame}

	\section{Outils}

		\subsection{Tri hybride}

			%\subsubsection{Algorithme}
			\begin{frame}
				\frametitle{Algorithme}

				On utilise un algorithme de tri rapide faisant appel à un
				tri par insertion pour les petits tableaux.
				\\~\\	
				\begin{algorithm}[H]
					\caption{Tri hybride du tableau t}
					\begin{algorithmic}[H]
						\IF{$taille(t) < minLength$} 
						\STATE{trier t avec un tri par insertion;} 
						\ELSE \STATE{
							prendre un pivot p;\\
							partitionner t selon le pivot p;\\
							trier récursivement les deux tableaux obtenus;
						}
						\ENDIF
					\end{algorithmic}
				\end{algorithm}

			\end{frame}
%%%%%%%%%%%%%%%%%%%%%%%%%%%%%%%%%%%%%%%%%%%%%%%%%%%%%%%%%%%%%%%%%%%%%%%%%%%%%%%
			%\subsubsection{Choix de minLength}
			\begin{frame}
				\frametitle{Choix de \emph{minLength}}
				
				Le choix est fait empiriquement après plusieurs tests.

				Insérer les graphes.

				On retient finalement la valeur $20$.

			\end{frame}
%%%%%%%%%%%%%%%%%%%%%%%%%%%%%%%%%%%%%%%%%%%%%%%%%%%%%%%%%%%%%%%%%%%%%%%%%%%%%%%
			%\subsubsection{Choix du pivot}
			\begin{frame}
				\frametitle{Choix du pivot}

				On implémente la méthode \emph{median of three} : 
				
				\begin{itemize}
					\item On considère les éléments de tête, queue et
						milieu du tableau à trier.
					\item On ordonne ces trois éléments dans le tableau.
					\item On prend la médiane de ces trois éléments comme
						pivot.
				\end{itemize}

			\end{frame}
%%%%%%%%%%%%%%%%%%%%%%%%%%%%%%%%%%%%%%%%%%%%%%%%%%%%%%%%%%%%%%%%%%%%%%%%%%%%%%%%
			\begin{frame}
				\frametitle{Choix du pivot}
				\framesubtitle{Comparaison des différents choix de pivot}

				Insérer les graphes.

			\end{frame}
%%%%%%%%%%%%%%%%%%%%%%%%%%%%%%%%%%%%%%%%%%%%%%%%%%%%%%%%%%%%%%%%%%%%%%%%%%%%%%%%
			%\subsubsection{Complexité}
			\begin{frame}
				\frametitle{Complexité}

				Insérer les graphes.

			\end{frame}
%%%%%%%%%%%%%%%%%%%%%%%%%%%%%%%%%%%%%%%%%%%%%%%%%%%%%%%%%%%%%%%%%%%%%%%%%%%%%%%%
		\subsection{Union-Find}
			\begin{frame}
				\frametitle{Présentation de la structure Union-Find}
		
			\end{frame}
%%%%%%%%%%%%%%%%%%%%%%%%%%%%%%%%%%%%%%%%%%%%%%%%%%%%%%%%%%%%%%%%%%%%%%%%%%%%%%%%
			\begin{frame}
				\frametitle{Première optimisation}

			\end{frame}
%%%%%%%%%%%%%%%%%%%%%%%%%%%%%%%%%%%%%%%%%%%%%%%%%%%%%%%%%%%%%%%%%%%%%%%%%%%%%%%%
			\begin{frame}
				\frametitle{Seconde optimisation}

			\end{frame}
%%%%%%%%%%%%%%%%%%%%%%%%%%%%%%%%%%%%%%%%%%%%%%%%%%%%%%%%%%%%%%%%%%%%%%%%%%%%%%%%
		\subsection{Générateur de graphes}
			\begin{frame}
				\frametitle{Contraintes du générateur}
				On doit créer un graphe :
				\begin{itemize}
					\item avec un nombre fixé de sommets
					\item avec un nombre fixé d'arêtes
					\item connexe
					\item uniformément distribué
					\item rapidement
				\end{itemize}
				Notre algorithme permet une telle génération en temps linéaire.
			\end{frame}
%%%%%%%%%%%%%%%%%%%%%%%%%%%%%%%%%%%%%%%%%%%%%%%%%%%%%%%%%%%%%%%%%%%%%%%%%%%%%%%%
			\begin{frame}
				\frametitle{Algorithme}
				
				\begin{algorithm}[H]
					\caption{Génération d'un graphe à n sommets
					et m arêtes}
					\begin{algorithmic}[H]
						\STATE $S,V \leftarrow \varnothing$
						\STATE $T \leftarrow \sigma\left\{1,...,n\right\}$
						\FORALL {sommet $i \in T$}
						\STATE $V \leftarrow V\cup\left\{(i,rand(S))\right\}$
						\STATE $S \leftarrow S\cup\left\{i\right\}$
						\STATE $T \leftarrow T\backslash\left\{i\right\}$
						\ENDFOR
						\STATE ajouter $m-n$ arêtes aléatoires à $V$
						\RETURN $(S,V)$
					\end{algorithmic}
				\end{algorithm}
				La complexité de cet algorithme est $\mathcal{O}(m)$
			\end{frame}

	\section{Algorithme de Kruskal}

	\subsection{Présentation}

	\subsection{Preuve de correction}

	\subsection{Complexité}

	\section{Voyageur de commence}

	\subsection{Algorithme}

	\subsection{Complexité}


\end{document}
