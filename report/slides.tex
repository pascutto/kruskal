\documentclass{beamer}

\usepackage[utf8]{inputenc}
\usepackage{algorithmic}
\usepackage{algorithm}
\usepackage[french]{babel}
\usepackage{tikz}
\usetikzlibrary{arrows,automata}
\usepackage{graphics}
\usepackage{pgfplots}

\usetheme{Warsaw}

\title[Algorithme de Kruskal]{Algorithme de Kruskal\\Application au 
problème du voyageur de commerce}
\author{Mihai Dusmanu, Clément Pascutto}
\institute{Algorithmique et programation\\École normale supérieure}
\date{Vendredi 8 Janvier 2016}


\begin{document}

    \begin{frame}
		\titlepage
    \end{frame}

	\begin{frame}
		\frametitle{Sommaire}
		\tableofcontents
	\end{frame}

	\section{Outils}

		\subsection{Tri hybride}

			%\subsubsection{Algorithme}
			\begin{frame}
				\frametitle{Algorithme}

				On utilise un algorithme de tri rapide faisant appel à un
				tri par insertion pour les petits tableaux. \pause
				\\~\\	
				\begin{algorithm}[H]
					\caption{Tri hybride du tableau t}
					\begin{algorithmic}[H]
						\IF{$taille(t) < minLength$} 
						\STATE{trier t avec un tri par insertion;} 
						\ELSE \STATE{
							prendre un pivot p;\\
							partitionner t selon le pivot p;\\
							trier récursivement les deux tableaux obtenus;
						}
						\ENDIF
					\end{algorithmic}
				\end{algorithm}

			\end{frame}
%%%%%%%%%%%%%%%%%%%%%%%%%%%%%%%%%%%%%%%%%%%%%%%%%%%%%%%%%%%%%%%%%%%%%%%%%%%%%%%
			%\subsubsection{Choix de minLength}
			\begin{frame}
				\frametitle{Choix de \emph{minLength}}
				
				Le choix est fait empiriquement après plusieurs tests.

				\begin{center}
\begin{tikzpicture}[scale=0.6]
\begin{axis}[
    title={Tests sur tableaux aléatoires de taille $1 000 000$},
    xlabel={minLength},
    ylabel={Temps d'éxécution [s]},
    xmin=0, xmax=50,
    ymin=0.17,ymax=0.21,
    xtick={0,10,20,30,40,50},
    ytick={0.17,0.18,0.19,0.20,0.21},
    legend pos=north west,
    ymajorgrids=true,
    grid style=dashed,
]
 
\addplot[
    color=blue,
	mark=square,
	mark size=0.7
    ]
    coordinates {
    (0,0.202960)(1,0.203280)(2,0.203410)(3,0.199960)(4,0.198010)(5,0.195420)(6,0.193360)(7,0.191530)(8,0.190440)(9,0.188560)(10,0.187710)(11,0.187440)(12,0.185680)(13,0.185140)(14,0.184200)(15,0.183820)(16,0.184020)(17,0.183480)(18,0.183140)(19,0.183080)(20,0.182550)(21,0.182860)(22,0.182680)(23,0.182990)(24,0.182690)(25,0.182700)(26,0.182890)(27,0.182860)(28,0.182660)(29,0.182900)(30,0.183060)(31,0.183280)(32,0.183200)(33,0.183500)(34,0.183570)(35,0.183620)(36,0.184300)(37,0.184180)(38,0.184620)(39,0.184720)(40,0.184770)(41,0.185150)(42,0.185370)(43,0.185820)(44,0.186130)(45,0.186730)(46,0.186350)(47,0.186650)(48,0.186810)(49,0.187270)(50,0.187770) 
	};
 
\end{axis}
\end{tikzpicture}
\end{center}
	
				\pause
				On retient finalement la valeur $20$.

			\end{frame}
%%%%%%%%%%%%%%%%%%%%%%%%%%%%%%%%%%%%%%%%%%%%%%%%%%%%%%%%%%%%%%%%%%%%%%%%%%%%%%%
			%\subsubsection{Choix du pivot}
			\begin{frame}
				\frametitle{Choix du pivot}

				On implémente la méthode \emph{median of three} : \pause 
				
				\begin{itemize}
					\item On considère les éléments de tête, queue et
						milieu du tableau à trier. \pause
					\item On ordonne ces trois éléments dans le tableau.
						\pause
					\item On prend la médiane de ces trois éléments comme
						pivot.
				\end{itemize}

			\end{frame}
%%%%%%%%%%%%%%%%%%%%%%%%%%%%%%%%%%%%%%%%%%%%%%%%%%%%%%%%%%%%%%%%%%%%%%%%%%%%%%%%
			\begin{frame}
				\frametitle{Choix du pivot}
				\framesubtitle{Comparaison des différents choix de pivot}

				\begin{center}
\begin{tikzpicture}[scale=0.8]
\begin{axis}[
    title={Tests de complexité selon le choix du pivot},
    xlabel={Taille du tableau},
    ylabel={Temps d'éxécution [s]},
    legend pos=north west,
    ymajorgrids=true,
    grid style=dashed,
	xmode=log,
	width=13cm,
	height=8cm
]
 
\addplot[color=red, mark=square, mark size=0.5]
    coordinates {
		(1000,0.000381)(2000,0.000795)(10000,0.003054)(20000,0.005002)(50000,0.012493)(100000,0.025005)(200000,0.053209)(500000,0.141692)(1000000,0.301091)(2000000,0.643868)
	};

\addplot[color=blue, mark=square, mark size=0.5]
	coordinates {
		(1000,0.000406)(2000,0.000901)(10000,0.003676)(20000,0.005117)(50000,0.012768)(100000,0.025317)(200000,0.054655)(500000,0.14572)(1000000,0.326452)(2000000,0.663702)
};

\addplot[color=green, mark=square, mark size=0.5]
	coordinates {
		(1000,0.000438)(2000,0.000881)(10000,0.003754)(20000,0.005347)(50000,0.013282)(100000,0.026137)(200000,0.055889)(500000,0.152488)(1000000,0.322487)(2000000,0.677414)
};

\addplot[color=black, mark=square, mark size=0.5]
	coordinates {
		(1000,0.000316)(2000,0.000703)(10000,0.005228)(20000,0.008305)(50000,0.021153)(100000,0.043995)(200000,0.093181)(500000,0.252968)(1000000,0.528827)(2000000,1.110281) 
	};

\legend{Median of three,Left,Random, C++ STL} 
\end{axis}
\end{tikzpicture}
\end{center}

			\end{frame}
%%%%%%%%%%%%%%%%%%%%%%%%%%%%%%%%%%%%%%%%%%%%%%%%%%%%%%%%%%%%%%%%%%%%%%%%%%%%%%%%
			\begin{frame}
				\frametitle{Complexité}
				
				\begin{center}
\begin{tikzpicture}[scale=0.8]
\begin{axis}[
    title={Tests de complexité de l'algorithme de tri hybride},
    xlabel={Taille du tableau},
    ylabel={$Temps/m\log m$},
    ymin=0,ymax=0.6,
    legend pos=north west,
    ymajorgrids=true,
    grid style=dashed,
	xmode=log,
	width=13cm,
	height=8cm
]
 
\addplot[
    color=blue,
	mark=square,
	mark size=0.7
    ]
    coordinates {
		(1000,0.382308094493256)(2000,0.362491172251114)(10000,0.177156152448253)(20000,0.175045516984215)(50000,0.1600677813884)(100000,0.150545100831557)(200000,0.151079177559705)(500000,0.149688600266954)(1000000,0.15106237070744)(2000000,0.153803093606565) 
	};
 
\end{axis}
\end{tikzpicture}
\end{center}

			\end{frame}

%%%%%%%%%%%%%%%%%%%%%%%%%%%%%%%%%%%%%%%%%%%%%%%%%%%%%%%%%%%%%%%%%%%%%%%%%%%%%%%%
		\subsection{Union-Find}
			\begin{frame}
				\frametitle{Présentation de la structure Union-Find}
				
				Union-find est une structure de données utilisée pour
				représenter une partition d'un ensemble. \pause
				
				Elle a trois primitives : \pause
				\begin{itemize}
					\item \texttt{new}($n$) crée une nouvelle structure
						Union-Find de taille $n$. \pause
					\item \texttt{find}($x$) renvoie le représentant de
						la classe de $x$. \pause
					\item \texttt{union}($x,y$) fusionne les classe de
						$x$ et $y$.
				\end{itemize}
			\end{frame}
%%%%%%%%%%%%%%%%%%%%%%%%%%%%%%%%%%%%%%%%%%%%%%%%%%%%%%%%%%%%%%%%%%%%%%%%%%%%%%%%
			\begin{frame}
				\frametitle{Présentation de la structure Union-Find}

				\begin{enumerate}
					\item \texttt{new(5)} 
						\hfill
						\begin{tikzpicture}
							\node (0) {0};
							\node (1) [right of=0] {1};
							\node (2) [right of=1] {2};
							\node (3) [right of=2] {3};
							\node (4) [right of=3] {4};
						\end{tikzpicture}
						\pause
					\item \texttt{union(0,3)}
						\hfill
						\begin{tikzpicture}[level distance=20pt]
							\node (0) {0}
								child{node{3}};
							\node (1) [right of=0] {1};
							\node (2) [right of=1] {2};
							\node (4) [right of=2] {4};
						\end{tikzpicture}
						\pause
					\item \texttt{union(3,2) \&\& union(1,4)}
						\hfill
						\begin{tikzpicture}[baseline=(3.base),level distance=20pt]
							\node (0) {0}
							child{node(3){3} 
									child{node{2}}
								};
							\node (1) [right of=0] {1}
								child{node{4}};
						\end{tikzpicture}
						\pause			
					\item \texttt{union(1,0)}
						\hfill
						\begin{tikzpicture}[baseline=(0.base),level distance=20pt]
							\node (1) {1}
							child{node(0){0}
									child{node{3} 
										child{node{2}}
									}
								}
								child{node{4}};
						\end{tikzpicture}
				\end{enumerate}

			\end{frame}
%%%%%%%%%%%%%%%%%%%%%%%%%%%%%%%%%%%%%%%%%%%%%%%%%%%%%%%%%%%%%%%%%%%%%%%%%%%%%%%%
			\begin{frame}
				\frametitle{Première optimisation}
				On modifie l'opération \texttt{union} pour réduire la
				profondeur de l'arbre.\pause
				\\~\\
				Exemple sur l'opération \texttt{union(3,2)} :
				\\~\\
				\begin{columns}[c]
					\begin{column}{0.46\textwidth}
						\begin{center}
						\begin{tikzpicture}[baseline=(0.base),level distance=20pt]
							\node (0) {0}
								child{node{3}};
							\node [right of=0] {2};
						\end{tikzpicture}
						$\longrightarrow$					
						\begin{tikzpicture}[baseline=(0.base),level distance=20pt]
							\node (0) {0}
							child{node(3){3} 
									child{node{2}}
								};
						\end{tikzpicture}
					\end{center}
					\end{column}
					\vrule{}
					\pause
					\begin{column}{0.46\textwidth}
						\begin{center}
						\begin{tikzpicture}[baseline=(0.base),level distance=20pt]
							\node (0) {0}
								child{node{3}};
							\node [right of=0] {2};
						\end{tikzpicture}
						$\longrightarrow$					
						\begin{tikzpicture}[baseline=(0.base),level distance=20pt]
							\node (0) {0}
								child{node{3}} 
								child{node{2}}
							;
						\end{tikzpicture}
						\end{center}
					\end{column}
			\end{columns}
			\pause
			~\\~\\
			C'est l'optimisation de compression	de chemins.
				
			\end{frame}
%%%%%%%%%%%%%%%%%%%%%%%%%%%%%%%%%%%%%%%%%%%%%%%%%%%%%%%%%%%%%%%%%%%%%%%%%%%%%%%%
			\begin{frame}
				\frametitle{Seconde optimisation}
				On modifie l'opération \texttt{union} pour réduire la profondeur
				de l'arbre.\pause
				\\~\\
				Exemple sur l'opération \texttt{union(1,0)} :
				\\~\\
				\begin{columns}[c]
					\begin{column}{0.46\textwidth}
						\begin{center}
						\begin{tikzpicture}[baseline=(0.base),level distance=20pt]
						\node (0) {0}
							child{node(3){3} 
									child{node{2}}
								};
							\node (1) [right of=0] {1}
								child{node{4}};
						\end{tikzpicture}
						$\longrightarrow$					
						\begin{tikzpicture}[baseline=(1.base),level distance=20pt]
							\node (1) {1}
							child{node(0){0}
									child{node{3} 
										child{node{2}}
									}
								}
								child{node{4}};
						\end{tikzpicture}
						\end{center}
					\end{column}
					\vrule{}
					\pause
					\begin{column}{0.46\textwidth}
						\begin{center}
						\begin{tikzpicture}[baseline=(0.base),level distance=20pt]
							\node (0) {0}
							child{node(3){3} 
									child{node{2}}
								};
							\node (1) [right of=0] {1}
								child{node{4}};
						\end{tikzpicture}
						$\longrightarrow$					
						\begin{tikzpicture}[baseline=(1.base),level distance=20pt]
							\node (1) {0}
							child{node{3} 
									child{node{2}}
							}
							child{node{1}
								child{node{4}}
							};
						\end{tikzpicture}
						\end{center}
					\end{column}
			\end{columns}

			\end{frame}
%%%%%%%%%%%%%%%%%%%%%%%%%%%%%%%%%%%%%%%%%%%%%%%%%%%%%%%%%%%%%%%%%%%%%%%%%%%%%%%%
			\begin{frame}
				\frametitle{Complexité en moyenne}

				Récapitulatif des complexités de Union-Find pour $m$
				opérations sur un structure de taille $n$ :
				\\~\\
				\begin{tabular}{|c|c|c|c|c|}
					\hline
					Primitive & $\varnothing$ & 1 & 2 & 1 \& 2
					\\
					\hline
					\texttt{new} & $\mathcal{O}(n)$ & $\mathcal{O}(n)$
					& $\mathcal{O}(n)$ & $\mathcal{O}(n)$
					\\
					\hline
					\texttt{find} & $\mathcal{O}(n)$ &
					$\mathcal{O}(\max(1,\frac
						{\log\frac{n^{2}}{m}}
						{\log\frac{2m}{n}}))$ &
					$\log(n)$ & $\alpha(m,n)$ 
					\\
					\hline
				\end{tabular}

			\end{frame}
%%%%%%%%%%%%%%%%%%%%%%%%%%%%%%%%%%%%%%%%%%%%%%%%%%%%%%%%%%%%%%%%%%%%%%%%%%%%%%%
			\begin{frame}
				\frametitle{Complexité}

				\begin{center}
\begin{tikzpicture}[scale=0.8]
\begin{axis}[
    title={Tests de complexité de Union-find},
    xlabel={Nombre d'opérations},
    ylabel={Temps d'éxécution [s]},
    legend pos=north west,
    ymajorgrids=true,
    grid style=dashed,
	xmode=log,
	width=13cm,
	height=8cm
]
 
\addplot[color=green, mark=square, mark size=0.5]
    coordinates {
	(500,0.000039)(1000,0.000177)(5000,0.000308)(10000,0.000771)(25000,0.001668)(50000,0.005205)(100000,0.008296)(250000,0.046119)(500000,0.081685)(1000000,0.134482)
	};

\addplot[color=red, mark=square, mark size=0.5]
	coordinates {
	(500,0.000141665084755749)(1000,0.000300447220984121)(5000,0.00169759905404294)(10000,0.00356046038570674)(25000,0.00944056057830622)(50000,0.0196874390773273)(100000,0.0409715088129778)(250000,0.107647201890499)(500000,0.223106396893394)(1000000,0.46170130783811)
};

\legend{Réel,Pire cas} 
\end{axis}
\end{tikzpicture}
\end{center}

			\end{frame}
%%%%%%%%%%%%%%%%%%%%%%%%%%%%%%%%%%%%%%%%%%%%%%%%%%%%%%%%%%%%%%%%%%%%%%%%%%%%%%%
		\subsection{Générateur de graphes}
			\begin{frame}
				\frametitle{Contraintes du générateur}
				On doit créer un graphe :
				\begin{itemize}
					\item avec un nombre fixé de sommets \pause
					\item avec un nombre fixé d'arêtes \pause
					\item connexe \pause
					\item uniformément distribué \pause
					\item rapidement \pause
				\end{itemize}
				Notre algorithme permet une telle génération en temps linéaire.
			\end{frame}
%%%%%%%%%%%%%%%%%%%%%%%%%%%%%%%%%%%%%%%%%%%%%%%%%%%%%%%%%%%%%%%%%%%%%%%%%%%%%%%%
			\begin{frame}
				\frametitle{Algorithme}
				
				\begin{algorithm}[H]
					\caption{Génération d'un graphe à n sommets
					et m arêtes}
					\begin{algorithmic}[H]
						\STATE $S,V \leftarrow \varnothing$
						\STATE $T \leftarrow \sigma\left\{1,...,n\right\}$
						\FOR {$i \in T$}
						\STATE $V \leftarrow V\cup\left\{(i,rand(S))\right\}$
						\STATE $S \leftarrow S\cup\left\{i\right\}$
						\STATE $T \leftarrow T\backslash\left\{i\right\}$
						\ENDFOR
						\STATE ajouter $m-n$ arêtes aléatoires à $V$
						\RETURN $(S,V)$
					\end{algorithmic}
				\end{algorithm}
				\pause
				La complexité de cet algorithme est $\mathcal{O}(m)$
			\end{frame}

	\section{Algorithme de Kruskal}
%%%%%%%%%%%%%%%%%%%%%%%%%%%%%%%%%%%%%%%%%%%%%%%%%%%%%%%%%%%%%%%%%%%%%%%%%%%%%%%%
	\subsection{Présentation}
		\begin{frame}
			\frametitle{Présentation de l'algorithme}
				
			\begin{algorithm}[H]
				\caption{Algorithme de Kruskal sur $G=(V,E)$}
				\begin{algorithmic}[H]
					\STATE $A \leftarrow \varnothing$
					\STATE trier $E$ par poids croissants
					\FOR {$(x,y,p)\in E$}
						\IF {$x$ et $y$ ne sont pas dans la même composante
						connexe}
						\STATE $A \leftarrow A \cup
						\left\{(x,y,p)\right\}$
						\ENDIF
					\ENDFOR
					\RETURN $(V,A)$
				\end{algorithmic}
			\end{algorithm}
		\end{frame}
%%%%%%%%%%%%%%%%%%%%%%%%%%%%%%%%%%%%%%%%%%%%%%%%%%%%%%%%%%%%%%%%%%%%%%%%%%%%%%%
		\begin{frame}
			\frametitle{Exemple d'éxécution}
			\begin{center}
			\begin{tikzpicture}[>=stealth',shorten >=1pt,auto,node
				distance=2.8cm,]
			
				\node[draw,circle] (1)					{$1$};
				\node[draw,circle] (2) [right=5cm]		{$2$};
				\node[draw,circle] (3) [above right of=2] {$3$};
				\node[draw,circle] (4) [below right of=2] {$4$};

				\path<1,2,3,4,5,6> (1) edge node {7} (2);
				\path<1,2,3>    (2) edge node {3}  (3);
				\path<1> (2) edge node {3}  (4);
				\path<1,2,3,4,5> (3) edge node {4}  (4);
				\path<1,2,3,4,5,6,7,8> (1) edge node {8}  (3);
				\path<1,2,3,4,5,6,7,8,9> (1) edge node {8}  (4);
				\path<2>[draw=red]	 (2) edge node {3} (4);
				\path<3,4,5,6,7,8,9,10>[draw=green,line width=1.5pt] (2) edge node {3} (4);
				\path<4>[draw=red]   (2) edge node {3} (3);
				\path<5,6,7,8,9,10>[draw=green,line width=1.5pt] (2) edge node {3} (3);
				\path<6,7,8,9,10>[draw=red]   (3) edge node {4} (4);
				\path<7>[draw=red]   (1) edge node {7} (2);
				\path<8,9,10>[draw=green,line width=1.5pt] (1) edge node {7} (2);
				\path<9,10>[draw=red] (1) edge node {8} (3);
				\path<10>[draw=red] (1) edge node {8} (4);
				\end{tikzpicture}
			\end{center}
		\end{frame}
%%%%%%%%%%%%%%%%%%%%%%%%%%%%%%%%%%%%%%%%%%%%%%%%%%%%%%%%%%%%%%%%%%%%%%%%%%%%%%%
	\subsection{Preuve de correction}
		\begin{frame}
			\frametitle{Preuve de correction}
			
			\begin{itemize}
				\item Le graphe renvoyé est clairement un arbre.
				\item On va montrer par induction qu'à tout instant de
					l'algorithme, $\mathcal{P}(A)$ : il existe un ACM
					contenant $A$. 
					\begin{itemize}
						\item $\mathcal{P}(\varnothing)$ est trivialement
							vraie.
						\item Supposons $T$ l'ACM garanti par
							$\mathcal{P}(A)$.\\
							Soit $e$ la prochaine arête.\\
							Si $e \in T$,
							$\mathcal{P}(A\cup\left\{e\right\})$
							est vraie.\\
							Sinon, $T+e$ contient un cycle $C$. Soit $f
							\in C \backslash A$.\\
							$T-f+e$ est un ACM qui contient
							$A\cup\left\{e\right\}$.
					\end{itemize}
			\end{itemize}
		\end{frame}
%%%%%%%%%%%%%%%%%%%%%%%%%%%%%%%%%%%%%%%%%%%%%%%%%%%%%%%%%%%%%%%%%%%%%%%%%%%%%%%
	\subsection{Complexité}
		\begin{frame}
			\frametitle{Complexité}
			
			\begin{center}
\begin{tikzpicture}[scale=0.8]
\begin{axis}[
    title={Tests de complexité de l'algorithme de Kruskal},
    xlabel={Nombre d'arêtes},
    ylabel={$Temps/m\log m$},
    ymin=0,ymax=1,
    legend pos=north west,
    ymajorgrids=true,
    grid style=dashed,
	xmode=log,
	width=13cm,
	height=8cm
]
 
\addplot[
    color=blue,
	mark=square,
	mark size=0.7
    ]
    coordinates {
		(500,0.530907263533321)(1000,0.506733826034368)(5000,0.536796418752264)(10000,0.620450563130947)(25000,0.614607869078732)(50000,0.579154914862072)(100000,0.555736629771359)(250000,0.584009143400593)(500000,0.566055578583395)(1000000,0.539472343212805)
	};
 
\end{axis}
\end{tikzpicture}
\end{center}
	
		\end{frame}
%%%%%%%%%%%%%%%%%%%%%%%%%%%%%%%%%%%%%%%%%%%%%%%%%%%%%%%%%%%%%%%%%%%%%%%%%%%%%%%
	\section{Voyageur de commence}

	\subsection{Algorithme}
		\begin{frame}
			\frametitle{Algorithme}
			
			\begin{algorithm}[H]
				\caption{build\_road($visited, node, edges$)}
				\begin{algorithmic}[H]
					\STATE $visited \leftarrow visited + node$
					\FOR{$(x,y,p)\in$ edges}
						\IF{$x = node$ \AND $y \ne father(node)$}
						\STATE build\_road($visited, y, edges$)
						\ELSE \IF{$y = node$ \AND $x \ne father(node)$}
						\STATE build\_road($visited, v, edges$)
						\ENDIF
						\ENDIF
					\ENDFOR
				\end{algorithmic}
			\end{algorithm}
			
		\end{frame}
%%%%%%%%%%%%%%%%%%%%%%%%%%%%%%%%%%%%%%%%%%%%%%%%%%%%%%%%%%%%%%%%%%%%%%%%%%%%%%%%
		\begin{frame}
			\frametitle{Exemple d'éxécution}

			\begin{center}
				\begin{tikzpicture}[>=stealth',shorten >=1pt,auto,node
				distance=2.8cm,]
			
				\node<1>[draw,circle] (1)					{$1$};
				\node<1,2>[draw,circle] (2) [right=5cm]		{$2$};
				\node<1,2,3,4>[draw,circle] (3) [above right of=2] {$3$};
				\node<1,2,3>[draw,circle] (4) [below right of=2] {$4$};
				\node<2,3,4,5,6>[draw=red,circle] (1)					{$1$};
				\node<3,4,5,6>[draw=red,circle] (2) [right=5cm]		{$2$};
				\node<5,6>[draw=red,circle] (3) [above right of=2] {$3$};
				\node<4,5,6>[draw=red,circle] (4) [below right of=2] {$4$};

				\path [draw=green] (2) edge node {3} (3);
				\path<1,2,3> [draw=green] (2) edge node {3} (4);
				\path<1,2,3,4> [draw]   (3) edge node {4} (4);
				\path<1,2> [draw=green] (1) edge node {7} (2);
				\path<1,2,3,4,5> [draw] (1) edge node {8} (3);
				\path [draw] (1) edge node {8} (4);
				
				\path<6> [draw=blue,line width=1.5pt] (1) edge node {8} (3);
				\path<5,6> [draw=blue,line width=1.5pt]   (3) edge node {4} (4);
				\path<3,4,5,6> [draw=blue,line width=1.5pt] (1) edge node {7} (2);
				\path<4,5,6> [draw=blue,line width=1.5pt] (2) edge node {3} (4);
				
			
				\end{tikzpicture}
			\end{center}

		\end{frame}
%%%%%%%%%%%%%%%%%%%%%%%%%%%%%%%%%%%%%%%%%%%%%%%%%%%%%%%%%%%%%%%%%%%%%%%%%%%%%%%%
	\begin{frame}
		\frametitle{Complexité}

		\begin{center}
\begin{tikzpicture}[scale=0.8]
\begin{axis}[
    title={Tests de complexité de l'algorithme du voyageur de commerce},
    xlabel={Nombre de sommets},
    ylabel={$Temps/n^{2}\log (n^{2})$},
    ymin=0,ymax=1.2,
    legend pos=north west,
    ymajorgrids=true,
    grid style=dashed,
	xmode=log,
	width=13cm,
	height=8cm
]
 
\addplot[
    color=blue,
	mark=square,
	mark size=0.7
    ]
    coordinates {
		(10,1.77607697441749)(25,0.489248569971374)(50,0.674715987076205)(75,0.727514442260125)(100,0.566688966837445)(250,0.463100665066087)(500,0.420567780829908)(750,0.418296086135812)(1000,0.428329560230366)(2500,0.417812913849976)
	};
 
\end{axis}
\end{tikzpicture}
\end{center}

	\end{frame}
%%%%%%%%%%%%%%%%%%%%%%%%%%%%%%%%%%%%%%%%%%%%%%%%%%%%%%%%%%%%%%%%%%%%%%%%%%%%%%%%

\end{document}
